\documentclass{article}
\author{Mark Workman}
\title{Post Project Writeup using \LaTeX}
\date{\today{}}
\begin{document}
\maketitle{}
\section{Project Summery}
\paragraph{}
We were tasked with writing a program that, given the PCAPs of network traffic involving this fell race, 
determine where the zerg are located on the battlefield.
The program should recommend a list of zerg to kill in order to form a fully-connected 
network of the survivors.
\section{Challenges}
\paragraph{}
The biggest problem I faced was doing the reseach for which type of R-Tree we were going to use.
The team took two days to research R-Tree's and decided to go with a quadTree after talking to the 
instructor. Getting all the points to be searchable in the tree was difficult at first. The next 
biggest challenge was getting the edges connected based on the lat and lon of the zerg being 
inserted into the tree. Once we got that working it was yet another challenge to try and connect edges 
based on altitude.
\section{Successes}
\paragraph{}
The biggest success of this project was being able to work as a team. Our team consisted of members that worked
very well together. It was very nice to have no finger pointing when problems in the code arose. The team sat down
and talked through everything.
\section{Lessons Learned}
\paragraph{}
The biggest lesson I learned was when we were finishing up the project. Having ample time to debug was key.
Stepping through valgrind in order to find errors or memory leaks and combating them one by one helped imensely.
\end{document}