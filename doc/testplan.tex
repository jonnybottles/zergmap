\documentclass[12pt]{article}
\usepackage{geometry}
\usepackage{hyperref}
\usepackage{listings}
\usepackage{dirtree}
\documentclass{article}
\usepackage{graphicx}
% page environment setup
\geometry{
    paper=letterpaper,
    margin=1.5cm,
    includefoot,
    footskip=30pt
}
\lstset{
    basicstyle=\ttfamily\footnotesize,
    breaklines=true,
}
\hypersetup{
    hidelinks,
    pdftitle={Ticker Test Plan},
    pdfauthor={Jonathan Butler},
    pdfdisplaydoctitle
}
\setlength{\parindent}{0pt}
% title setup
\title{Ticker Test Plan}
\author{Jonathan Butler, Adam Matthes, Mark Workman}
\date{February 1\textsuperscript{st}, 2022}
\begin{document}
\maketitle
\section{Introduction}

 This program allows for the user to pass multiple \lstinline|.pcap| files via command line As such, there are a number of sample files for testing found in the project under the test folder.

\section{Sample Files}
\subsection{Non-Conforming Map Files}
\hspace{8pt}
\begin{enumerate}
    \item \lstinline|test/samp/empty.pcap:
        Empty file.
    \item \lstinline|test/samp/gmailconvert.pcap|:
        Non-Zerg file containing Gmail network traffic.
\end{enumerate}

\subsection{Instructor Provided Map Files}
\begin{enumerate}
    \item \lstinline|test/samp/C_getstatus.pcap|
    \item \lstinline|test/samp/C_getstatus_S_normal.pcap|
    \item \lstinline|test/samp/G_normal.pcap|
    \item \lstinline|test/samp/S_normal.pcap|
    \item \lstinline|test/samp/M_hello.pcap|
    \item \lstinline|test/samp/M_ipv6.pcap|
    \item \lstinline|test/samp/bad_multi_command.pcap|
    \item \lstinline|test/samp/easy_3n0r/packet00.pcap|
    \item \lstinline|test/samp/easy_3n0r/packet01.pcap|
    \item \lstinline|test/samp/easy_3n0r/packet02.pcap|
    \item \lstinline|test/samp/easy_3n1r/packet00.pcap|
    \item \lstinline|test/samp/easy_3n1r/packet01.pcap|
    \item \lstinline|test/samp/easy_3n1r/packet02.pcap|
    \item \lstinline|test/samp/easy_5nxx/packet00.pcap|
    \item \lstinline|test/samp/easy_5nxx/packet01.pcap|
    \item \lstinline|test/samp/easy_5nxx/packet02.pcap|
    \item \lstinline|test/samp/easy_5nxx/packet03.pcap|
    \item \lstinline|test/samp/easy_5nxx/packet04.pcap|
    \item \lstinline|test/samp/easy_v62n0r/packet00.pcap|
    \item \lstinline|test/samp/easy_v62n0r/packet01.pcap|
    \item \lstinline|test/samp/med_4n1r/packet00.pcap|
    \item \lstinline|test/samp/med_4n1r/packet01.pcap|
    \item \lstinline|test/samp/med_4n1r/packet02.pcap|
    \item \lstinline|test/samp/med_4n1r/packet03.pcap|
    \item \lstinline|test/samp/hard_6n0r/packet00.pcap|
    \item \lstinline|test/samp/hard_6n0r/packet01.pcap|
    \item \lstinline|test/samp/hard_6n0r/packet02.pcap|
    \item \lstinline|test/samp/hard_6n0r/packet03.pcap|
    \item \lstinline|test/samp/hard_6n0r/packet04.pcap|
    \item \lstinline|test/samp/hard_6n0r/packet05.pcap|
\end{enumerate}

\subsection{Custom Built PCAP Files}
\begin{enumerate}
    \item \lstinline|test/samp/multimerge.pcap|:
        Multiple instructor provided \lstinline|.pcap| files combined into one.
    
\end{enumerate}

\section{Automated Test Cases}
Automated test cases may be executed with make \lstinline|make check|. This will both build the tests and run them. These are the unit tests of functions.

\subsection{Common Test Cases}

Add a build target of make profile, which builds the program with profiling information.

\hspace{8pt}
\begin{enumerate}
    \item Installation
    \newline
        \lstinline|git clone git@git.umbc.tc:tdqc/tdqc10/jbutler/zergmap.git; cd zergmap|
        \newline
            \textbf{Expected}: Directory is created.
            \newline
            \textbf{Perquisites}: In the project's directory.
    \item Correct Branch
    \newline
        \lstinline|git branch|
        \newline
            \textbf{Expected}: \lstinline|master| is active.
    \item Build Cleaning
    \newline
        \lstinline|make; make clean; ls *.o|
        \newline
            \textbf{Expected}: No such file or directory error.
\end{enumerate}

\subsection{Zergmap Test Cases}

\hspace{8pt}
\begin{enumerate}
    \item Build
    \newline
        \lstinline|make zergmap; ls zergmap|
        \newline
            \textbf{Expected}: file is listed.
    \item Build
    \newline
        \lstinline|make; ls zergmap|
        \newline
            \textbf{Expected}: file is listed
    \item Build Debugging
    \newline
        \lstinline|make debug; readelf --debug-dump=decodedline zergmap|
        \newline
            \textbf{Expected}: Debugging info is listed.
    \item No Command-Line Arguments
    \newline
        \lstinline|./zergmap|
        \newline
            \textbf{Expected}: Program exits with usage error.
    \item Non-Existent File
    \newline
        \lstinline|./zergmap test/samp/doesnotexist.pcap|
        \newline
            \textbf{Expected}: Program exits with "no such file" error.
        \item Multiple PCAPs One Good One Bad
    \newline
        \lstinline|./zergmap test/samp/bad_multi_command.pcap test/samp/G_normal.pcap|
        \newline
            \textbf{Expected}: Program processes both files and prints out number \lstinline|.pcap| files that were (1) and were not (1) able to be processed.
    \item Empty \lstinline|.pcap| File
    \newline
        \lstinline|./zergmap/test/samp/empty.pcap|
        \newline
            \textbf{Expected}: Program exits with cannot ready all file heard bytes error.
    \item Gmail File
    \newline
        \lstinline|./zergmap test/samp/gmailconvert.pcap|
        \newline
            \textbf{Expected}: Program exits with multiple "unable to decode" errors.
    \item Hello Message File
    \newline
        \lstinline|./zergmap test/samp/M_hello.pcap|
        \newline
            \textbf{Expected}: Program displays "zergmap does not process message payloads" message.
    \item Command Get Status File
    \newline
        \lstinline|./zergmap test/samp/C_getstatus.pcap; |
        \newline
            \textbf{Expected}: Program exits normally with GET STATUS command.
    \item Command Get Status Status Normal File
    \newline
        \lstinline|./zergmap test/samp/C_getstatus_S_normal.pcap; |
        \newline
            \textbf{Expected}: Program exits normally with no message printed, as health is above 10%.
    \item GPS Normal File
    \newline
        \lstinline|./zergmap /test/samp/G_normal.pcap|
        \newline
            \textbf{Expected}: Program exits with normally with GPS coordinates message. 
    \item Status Normal File
    \newline
        \lstinline|./zergmap -h 75 /test/samp/S_normal.pcap|
        \newline
            \textbf{Expected}: Program exits normally and prints "low health" status message.
    \newline
        \lstinline|./zergmap /test/samp/bad_multi_command.pcap|
        \newline
            \textbf{Expected}: Program exits with normally with four "bad packet" messages. 
    \item Message IPv6 File
    \newline
        \lstinline|./zergmap /test/samp/M_ipv6.pcap|
        \newline
            \textbf{Expected}: Program exits normally and displays "zergmap does not process message payloads" message.
    \item Zergmap Instructor Provided Files
    \newline
        \lstinline|./zergmap /test/samp/<dirname>/<filename>.pcap|
        \newline
            \textbf{Expected}: Program processes \lstinline|.pcap file| and prints health status accordingly and recommends Zerg(s) for removal if any.
\end{enumerate}

\end{document}
